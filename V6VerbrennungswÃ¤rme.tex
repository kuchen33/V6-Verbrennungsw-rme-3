
\documentclass[12pt,a4paper,titlepage,headinclude,bibtotoc]{scrartcl}

%---- Allgemeine Layout Einstellungen ------------------------------------------

% Für Kopf und Fußzeilen, siehe auch KOMA-Skript Doku
\usepackage[komastyle]{scrpage2}
\pagestyle{plain}
\setheadsepline{0.5pt}[\color{black}]
\automark[section]{chapter}


%Einstellungen für Figuren- und Tabellenbeschriftungen
\setkomafont{captionlabel}{\sffamily\bfseries}
\setcapindent{0em}


%---- Weitere Pakete -----------------------------------------------------------
% Die Pakete sind alle in der TeX Live Distribution enthalten. Wichtige Adressen
% www.ctan.org, www.dante.de

% Sprachunterstützung
\usepackage[ngerman]{babel}

% Benutzung von Umlauten direkt im Text
% entweder "latin1" oder "utf8"
\usepackage[utf8]{inputenc}

% Pakete mit Mathesymbolen und zur Beseitigung von Schwächen der Mathe-Umgebung
\usepackage{latexsym,exscale,stmaryrd,amssymb,amsmath}


\usepackage[nointegrals]{wasysym}
\usepackage{eurosym}

% Anderes Literaturverzeichnisformat
%\usepackage[square,sort&compress]{natbib}
\usepackage{hyperref}
% Für Farbe
\usepackage{color}
\usepackage{graphicx}
\usepackage{wrapfig}
\usepackage{subfigure}

% Caption neben Abbildung
\usepackage{sidecap}


% Befehl für "Entspricht"-Zeichen
\newcommand{\corresponds}{\ensuremath{\mathrel{\widehat{=}}}}
% Befehl für Errorfunction
\newcommand{\erf}[1]{\text{ erf}\ensuremath{\left( #1 \right)}}


%Fußnoten zwingend auf diese Seite setzen
\interfootnotelinepenalty=1000

%Für chemische Formeln (von www.dante.de)
%% Anpassung an LaTeX(2e) von Bernd Raichle
\makeatletter
\DeclareRobustCommand{\chemical}[1]{%
  {\(\m@th
   \edef\resetfontdimens{\noexpand\)%
       \fontdimen16\textfont2=\the\fontdimen16\textfont2
       \fontdimen17\textfont2=\the\fontdimen17\textfont2\relax}%
   \fontdimen16\textfont2=2.7pt \fontdimen17\textfont2=2.7pt
   \mathrm{#1}%
   \resetfontdimens}}
\makeatother
\usepackage{textcomp}
\usepackage{upgreek}
%\begin{document}
%$\upmu$
%\end{document}
%Honecker-Kasten mit $$\shadowbox{$xxxx$}$$
\usepackage{fancybox}

%SI-Package
\usepackage{siunitx}

%keine Einrückung, wenn Latex doppelte Leerzeile
\parindent0pt

%Bibliography \bibliography{literatur} und \cite{gerthsen}
%\usepackage{cite}
\usepackage{babelbib}
\selectbiblanguage{ngerman}

\usepackage{siunitx}
%\begin{document}
 % \SI{1.55}{\micro\metre}
\sisetup{math-micro=\text{µ},text-micro=µ}
\usepackage{amsmath}

\usepackage{siunitx}
%\begin{document}
 % \SI{1.55}{\micro\metre}
\sisetup{math-micro=\text{µ},text-micro=µ}
\usepackage{amsmath}
\usepackage[verbose]{placeins}
\usepackage{setspace}
\usepackage{threeparttable}
\usepackage[verbose]{placeins}


\begin{document}

\begin{titlepage}
\centering
\textsc{\Large Physikalisch- Chemisches Grundpraktikum\\[1.5ex] Universität Göttingen}

\vspace*{0.5cm}

\rule{\textwidth}{1pt}\\[0.5cm]
{\huge \bfseries
  Versuch 6: \\[1.5ex]
  Verbrennungswärme einer festen organischen Substanz}\\[0.5cm]
\rule{\textwidth}{1pt}

\vspace*{0.5cm}


\begin{Large}
\begin{tabular}{ll}
Durchführende: &  Isaac Maksso, Julia Stachowiak\\
Assistent: & Jannis Neugebohren\\
Versuchsdatum: & 08.12.2016\\
Datum der ersten Abgabe: & 15.12.2016\\
\end{tabular}
\end{Large}

\vspace*{0.5cm}




\vspace{1.3cm} 
\end{titlepage}


\tableofcontents %=Inhaltsverzeichnis

\section{Experimentelles}

\subsection{Versuchsaufbau}
\subsection{Durchführung}
Die Verbrennungswärme von Naphtalin sollte mittels eines Kalorimeters mit Berthelot-Mahlerscher Bombe ermittelt werden.\\
Zur Kalibrierung des Thermoelements wurden ca. 0,6 g mit einem vorher gedrehten und genau gewogenen Nickeldraht in eine Tablette gepresst. An einer Stelle wurde vorher eine Spur des doppelt gewickelten Drahtes durchtrennt und so gelegt, dass sich diese Zündungsstelle innerhalb der Tablette befand. Die Tablette wurde am Nickeldraht an zwei Elektroden befestigt und die Bombe angeschraubt. Anschließend wurde diese mit 25 atm O$_2$ befüllt und in das Wasserbad gestellt.  Die Temperaturmessung erfolgte über einen Pt1000-\,Temperaturmesskopf (Verstärkung mit einem Pt1000-Vorverstärker um 20mV/K). Nach ca. 3 Minuten Vorperiode wurde die Probe gezündet und ca. 5 Minuten lang weitergemessen(Nachperiode).
Die Temperaturaufzeichung erfolgte mittels LabView.\\
Der Vorgang wurde mit Bezoesäure(zur Ermittlung der Wärmekapazität des Thermoelements) und Naphtalin je drei mal durchgeführt.\\ 

\section{Auswertung}

Für die Verbrennung von Naphtalin (\ref{Naphtalin}) und Benzoesäure (\ref{Benzoesaeure}) ergeben sich folgende Reaktionsgleichungen:\\

\begin{equation}\label{Benzoesaeure}
\mathrm{C_6}\mathrm{H}_5\mathrm{COOH}_\mathrm{(s)} + 7,5\,{\mathrm{O}_2}_\mathrm{(g)} \rightarrow 7\,{\mathrm{CO}_2}_\mathrm{(g)} +3\,\mathrm{H}_2\mathrm{O}_\mathrm{(l)}
\end{equation}

\begin{equation}\label{Naphtalin}
{\mathrm{C}_{10}\mathrm{H}_8}_\mathrm{(s)} + 12\,{\mathrm{O}_2}_\mathrm{(g)} \rightarrow 10\,{\mathrm{CO}_2}_\mathrm{(g)} + 4\,\mathrm{H}_2\mathrm{O}_\mathrm{(l)}
\end{equation}

Nach dem Satz von Hess kann die Reaktionsenthalpie aus der Differenz der Standardbildungsenthalpien $\Delta H_\mathrm{f}^0$ zwischen Produkten und Edukten berechnet werden:\\

\begin{equation} \label{GleichungHess}
\Delta H_\mathrm{C} = \sum_i \nu_i\Delta H_{\mathrm{f},i}\mathrm{(Produkte)} - \sum_i \nu_i\Delta H_{\mathrm{f},i}\mathrm{(Edukte)}
\end{equation}

Die Literaturwerte für die molaren Standardbildungsenthalpien und die nach Gleichung (\ref{GleichungHess}) berechneten Verbrennungsenthalpien $\Delta H_\mathrm{C,m}$ für Benzoesäureund Naphtalin sind in  Tabelle(\ref{TABELLELiteraturwerte Bildungsenthalpien}) aufgelistet.\\

\begin{center}
\begin{table} \label{TABELLELiteraturwerte Bildungsenthalpien}
\caption{molare Standardbildungsenthalpien sowie Verbrennungsenthalpien für Benzoesäure und Naphtalin}
\begin{tabular}{c|c|c}
  & $\Delta H_{\mathrm{f,m}}$ [kJ$\cdot$mol$^{-1}$] &$\Delta H_\mathrm{C,m}$[kJ$\cdot$mol$^{-1}$]\\ 
 \hline 
 $\mathrm{C}_6\mathrm{H}_5\mathrm{COOH}$ & -384,8$^2$&-5156 \\ 
 \hline 
 $\mathrm{C}_{10}\mathrm{H}_8$ & 78,5$^1$ &3227\\
 \hline 
 $\mathrm{O}_2\mathrm{(g)}$ & 0 \\ 
 \hline 
 $\mathrm{CO}_2\mathrm{g}$ & -393,51$^1$ \\ 
 \hline 
 $\mathrm{H}_2\mathrm{O}_\mathrm{l}$ & –285.830$^1$ \\ 
 \end{tabular}  
\end{table}
\end{center}
\FloatBarrier

\subsection{Bestimmung der Wärmekapazität des Kalorimeters und der Änderung der inneren Energie}
Die Wärmekapazität bei konstantem Volumen ist folgendermaßen definiert:\\

\begin{equation}\label{Wärmekapazität}
C_v=\left(\frac{\delta Q}{\partial T}\right)_v =\left(\frac{\partial U}{\partial T}\right)_v
\end{equation}

Integration liefert einen Differenzenquotienten. Die aus den Auftragungen ermittelten Temperaturdifferenzen sind in Tabelle (\ref{TabDeltaT}) dargestellt. 
\FloatBarrier

\begin{table} \label{TabDeltaT}\caption{Aus den Auftragungen ermittelte Temperaturdifferenzen.}
\begin{tabular}{c|c|c|c|c|c|c}
&\multicolumn{3}{c|}{Benzoesäure} & \multicolumn{3}{c}{Naphtalin}\\ 
Messung& 1&2&3&1&2&3\\
\hline 
$\Delta T$ /K & 1,21 & 1,17 & 1,15 & 2,31 & 1,88 & 2,05 \\ 
$m$ /g&0,6460&0,6044&0,6208&0,8229&0,6166&0,7072\\
$n$ /mmol& 5,288&4,949&5,084&6,420&4,811&5,518\\
\end{tabular} 
\end{table}
\FloatBarrier

Das totale Differential der inneren Energie lautet folgendermaßen:\\

\begin{equation}
\mathrm{d}U= T\mathrm{d}S - p\mathrm{d}V
\end{equation} 

Bei konstantem Volumen fällt der letzte Term weg und folgender Zusammenhang mit der Enthalpie besteht:\\

\begin{equation} \label{Gl.6.1}
\mathrm{d}H= T\mathrm{d}S + V\mathrm{d}p = \mathrm{d}U + V\mathrm{d}p = \Delta U +\mathrm{R}T_0\sum_i \nu_{i,gas}
\end{equation}

\textbf{Wie man auf letzten Term der Gleichung kommt}

Für Benzoesäure kann die Wärmekapazität nach Gleichung (\ref{Wärmekapazität}) und die Änderung der inneren Energie nach Gleichung (\ref{Gl.6.1}) bestimmt werden. $T_0$ beschreibt dabei die Anfangstemperatur der Messung. Nach der Reaktionsgleichung für die Verbrennung von Benzoesäure ergibt sich $\sum \nu_{i,\mathrm{gas}}= -0,5$. Die Ergebnisse sind in Tabelle (\ref{TabDeltaT}) dargestellt.\\

\begin{table} \label{TabDeltaT}\caption{Caption Werte für Benzoesäure}
\begin{tabular}{c|c|c|c}
Messung& 1&2&3\\
\hline 
$T_0$ /K&294,584&295,909&301,017\\
$\Delta U$ /kJ$\cdot \mathrm{mol}^{-1} \cdot \mathrm{K}^{-1}$&
-3225,8&-3225,8&-3225,7\\
$c_{v,\mathrm{m}}$ /kJ$\cdot \mathrm{mol}^{-1} \cdot \mathrm{K}^{-1}$&-14,102&-13,645&-14,259\\
\end{tabular} 
\end{table}
\FloatBarrier
-------------------------------\\


\subsection{Bestimmung der Verbrennungswärme $\Delta H_\mathrm{C,m}$ von Naphtalin}

Mit der bestimmten molaren Wärmekapazität des Thermoelements Kann die Verbrennungswärme von Naphtalin berechnet werden. Einsetzen in Glichung (\ref{Gl.6.1}) mit $\sum_i \nu_i$= -2 liefert die Verbrennungsenthalpie $\Delta H_\mathrm{C}$.\\

\begin{equation}
\Delta U= \Delta Q= C_{v,\mathrm{m}}\cdot \Delta T
\end{equation}

Zur Berechnung wurde der Wert der 3. Messung für $C_{v,\mathrm{m}}$ verwendet, da die Anfangstemperaturen der Naphtalin-Messungen der Anfangstemperatur der 3. Messung am nächsten kommen. Die Ergebnisse sind in Tabelle (\ref{Endergebnisse}) zu sehen.\\

\begin{table} \caption{CAPTION Endgergebnisse} \label{Endergebnisse}
\begin{tabular}{c|c|c|c}
Messung & 1 & 2 & 3 \\ 
\hline 
$T_0$ /K &298,14&300,2&300,9\\
\hline 
$\Delta U$ /J$\cdot$mol$^{-1}\cdot$K &-32,939&-26,807&-29,232\\ 
\hline 
$\Delta H_\mathrm{m}$ /J$\cdot$mol$^{-1}\cdot$K &-37,897&-31,799&-34,235\\
\end{tabular} 
\end{table}
\FloatBarrier

\subsection{Fehlerrechnung}
Folgende fehlerbehaftete Größen traten während des Versuchs auf:\\
 $\Delta m$= 0,00005 g\\
 $\Delta T_0$= 0,1 K\\
 
Die Fehler der Temperaturdifferenzen ergeben sich aus Grenzgeraden der Auftragungen: $\Delta \Delta T= \frac{\Delta T_\mathrm{max}-\Delta T_\mathrm{min}}{2} $.\\

\textbf{Tabelle mit $\Delta \Delta T$ für jede Auftragung.}\\

Zur Fehlerbestimmung wird die Größtfehlerfortpflanzung verwendet:\\

\begin{equation}
\Delta f=\sum_i \left| \left(\frac{\partial f}{\partial x_i}\right)\right| \cdot \Delta x_i
\end{equation}

Die Fehler für $C_{v,\mathrm{m}}$ und $\Delta H_\mathrm{m}$ ergeben sich folgendermaßen:\\

\begin{equation}
\Delta C_{v,\mathrm{m}}= \left|-\frac{\Delta H+0,5 \cdot \mathrm{R}\cdot T_0 \cdot n}{\Delta T^2}\right|\cdot \Delta \Delta T + \left|-\frac{\Delta H+0,5 \cdot \mathrm{R} \cdot n}{\Delta T}\right| \cdot \Delta T_0+ \left|-\frac{\Delta H+0,5 \cdot \mathrm{R}\cdot T_0}{\Delta T}\right|\cdot \Delta n
\end{equation}

\begin{equation}
\Delta \Delta H_\mathrm{m}= \left|\frac{1}{\Delta T}\right| \cdot \Delta C_{v,\mathrm{m}}+ \left|-\frac{C_{v,\mathrm{m}}}{\Delta T^2}\right|\cdot \Delta \Delta T+\left|-2\mathrm{R}\right| \cdot \Delta T
\end{equation}

\textbf{Tabelle mit je Fehlern für $C_v$ und Delta H}.\\

\section{Fehlerdiskussion}
\textbf{in Tabelle darstellen: alle Ergebnisse inkl. Fehler und Literaturwerte}\\

gerechnet mit Standardbildungsenthalpien; heißt bei Standardbedingungen 25 Grad die wir aber nicht hatten\\

Bei der Berechnung der Verbrennungswärme wurde die Wärmekapazität des Kalorimeters bei einer leicht anderen Temperatur verwendet.\\

\section{Anhang}
Auftragungen


\section{Literaturverzeichnis}

1\quad \emph{CRC Handbook of Chemistry and Physics}, 84. Auflage; D.R. Lide; CRC Press LLC: Boca Raton, \textbf{2004}.
\vspace{0,5 cm}


2 \quad \url{http://webbook.nist.gov/cgi/cbook.cgi?ID=C65850&Mask=2}; aufgerufen am 10.12.2016.\\
\vspace{0,5 cm}

%2\quad Kabo, G.J.; Kozyro,A., A.; Frenkel, M.; Blokhin, A. V.\emph{Mol. Cryst. Liq. Cryst.}, \textbf{1999}, \emph{326}, 333-335.

%5\quad Eckhold, Götz: \emph{Praktikum I zur Physikalischen Chemie}, Institut für Physikalische Chemie, Uni Göttingen, \textbf{2014}.

%\vspace{0,5 cm}

%6 \quad Eckhold, Götz: \emph{Statistische Thermodynamik}, Institut für Physikalische Chemie, Uni Göttingen, \textbf{2012}.

%\vspace{0,5cm}

%7 \quad Eckhold, Götz: \emph{Chemisches Gleichgewicht}, Institut für Physikalische Chemie, Uni Göttingen, \textbf{2015}.\\

%\vspace{0,5cm}

%8 \quad Atkins, P.W.: \emph{Physikalische Chemie}, Wiley-VCH, Weinheim, \textbf{2006}.\\

%\vspace{0,5cm}

%9 \quad Zemansky: \emph{Heat and Thermodynamics},Mc Graw-Hill, New York, \textbf{1990}.\\

\end{document}
