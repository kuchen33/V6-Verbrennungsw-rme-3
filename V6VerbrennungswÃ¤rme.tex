
\documentclass[12pt,a4paper,titlepage,headinclude,bibtotoc]{scrartcl}

%---- Allgemeine Layout Einstellungen ------------------------------------------

% Für Kopf und Fußzeilen, siehe auch KOMA-Skript Doku
\usepackage[komastyle]{scrpage2}
\pagestyle{plain}
\setheadsepline{0.5pt}[\color{black}]
\automark[section]{chapter}


%Einstellungen für Figuren- und Tabellenbeschriftungen
\setkomafont{captionlabel}{\sffamily\bfseries}
\setcapindent{0em}


%---- Weitere Pakete -----------------------------------------------------------
% Die Pakete sind alle in der TeX Live Distribution enthalten. Wichtige Adressen
% www.ctan.org, www.dante.de

% Sprachunterstützung
\usepackage[ngerman]{babel}

% Benutzung von Umlauten direkt im Text
% entweder "latin1" oder "utf8"
\usepackage[utf8]{inputenc}

% Pakete mit Mathesymbolen und zur Beseitigung von Schwächen der Mathe-Umgebung
\usepackage{latexsym,exscale,stmaryrd,amssymb,amsmath}


\usepackage[nointegrals]{wasysym}
\usepackage{eurosym}

% Anderes Literaturverzeichnisformat
%\usepackage[square,sort&compress]{natbib}
\usepackage{hyperref}
% Für Farbe
\usepackage{color}
\usepackage{graphicx}
\usepackage{wrapfig}
\usepackage{subfigure}

% Caption neben Abbildung
\usepackage{sidecap}


% Befehl für "Entspricht"-Zeichen
\newcommand{\corresponds}{\ensuremath{\mathrel{\widehat{=}}}}
% Befehl für Errorfunction
\newcommand{\erf}[1]{\text{ erf}\ensuremath{\left( #1 \right)}}


%Fußnoten zwingend auf diese Seite setzen
\interfootnotelinepenalty=1000

%Für chemische Formeln (von www.dante.de)
%% Anpassung an LaTeX(2e) von Bernd Raichle
\makeatletter
\DeclareRobustCommand{\chemical}[1]{%
  {\(\m@th
   \edef\resetfontdimens{\noexpand\)%
       \fontdimen16\textfont2=\the\fontdimen16\textfont2
       \fontdimen17\textfont2=\the\fontdimen17\textfont2\relax}%
   \fontdimen16\textfont2=2.7pt \fontdimen17\textfont2=2.7pt
   \mathrm{#1}%
   \resetfontdimens}}
\makeatother
\usepackage{textcomp}
\usepackage{upgreek}
%\begin{document}
%$\upmu$
%\end{document}
%Honecker-Kasten mit $$\shadowbox{$xxxx$}$$
\usepackage{fancybox}

%SI-Package
\usepackage{siunitx}

%keine Einrückung, wenn Latex doppelte Leerzeile
\parindent0pt

%Bibliography \bibliography{literatur} und \cite{gerthsen}
%\usepackage{cite}
\usepackage{babelbib}
\selectbiblanguage{ngerman}

\usepackage{siunitx}
%\begin{document}
 % \SI{1.55}{\micro\metre}
\sisetup{math-micro=\text{µ},text-micro=µ}
\usepackage{amsmath}

\usepackage{siunitx}
%\begin{document}
 % \SI{1.55}{\micro\metre}
\sisetup{math-micro=\text{µ},text-micro=µ}
\usepackage{amsmath}
\usepackage[verbose]{placeins}
\usepackage{setspace}
\usepackage{threeparttable}
\usepackage[verbose]{placeins}


\begin{document}

\begin{titlepage}
\centering
\textsc{\Large Physikalisch- Chemisches Grundpraktikum\\[1.5ex] Universität Göttingen}

\vspace*{0.5cm}

\rule{\textwidth}{1pt}\\[0.5cm]
{\huge \bfseries
  Versuch 6: \\[1.5ex]
  Verbrennungswärme einer festen organischen Substanz}\\[0.5cm]
\rule{\textwidth}{1pt}

\vspace*{0.5cm}


\begin{Large}
\begin{tabular}{ll}
Durchführende: &  Isaac Maksso, Julia Stachowiak\\
Assistent: & Jannis \\
Versuchsdatum: & 08.12.2016\\
Datum der ersten Abgabe: & 15.12.2016\\
\end{tabular}
\end{Large}

\vspace*{0.5cm}




\vspace{1.3cm} 
\end{titlepage}


\tableofcontents %=Inhaltsverzeichnis

\section{Experimentelles}

\subsection{Versuchsaufbau}
\subsection{Durchführung}
Die Verbrennungswärme einer festen unbekannten Substanz sollte mittels eines Kalorimeters mit Berthelot-Mahlerscher Bombe ermittelt werden.
Dabei wurde je ein Temperatur-Zeit-Diagramm mit LabView aufgestellt und die anderen Größen aus der Temperaturdifferenz bestimmt. Die Temperaturmessung erfolgte über einen Pt1000-\,Temperaturmesskopf (Verstärkung mit einem Pt1000-Vorverstärker um 20mV/K). Die Vor- und Nachperiode \textbf{betrug jeweils 5-7 Minuten}.\\
Zur Bestimmung der Wärmekapazität des Kalorimeters wurden ca. 0,6 g Benzoesäure sowie ein vorher gewogener Nickeldraht zur Zündung in eine Tablette gepresst. Diese wurde anschließend mit 20-25 atm O$_2$ befüllt und die Enden des Nickeldrahtes an die Elektroden angeschlossen und die Temperatur von LabView aufgezeichnet.\\
Anschließend wurde der Vorgang mit der organischen Substanz wiederholt. Der Versuch wurde jeweils 3 Mal wiederholt. Als erste organische Substanz diente Naphtalin und anschließend die unbekannte Substanz. Ein Durchlauf wurde mit \textbf{selbst mitgebrachte Substanz} durchgeführt.\\

\section{Auswertung}

Für die Verbrennung von Naphtalin (\ref{Naphtalin}) und Benzoesäure %(\ref{Benzoesäure}) ergeben sich folgende Reaktionsgleichungen:\\

%\begin{equation}\label{Benzoesäure}
%2\mathrm{C}_6\mathrm{H}_5\mathrm{COOH} + 15\mathrm{O}_2 \rightarrow 14\mathrm{CO}_2 + 6\mathrm{H}_2\mathrm{O}
%\end{equation}


\begin{equation}\label{Naphtalin}
\mathrm{C}_{10}\mathrm{H}_8 + 15\mathrm{O}_2 \rightarrow 14 \mathrm{CO}_2 + 6\mathrm{H}_2\mathrm{O}
\end{equation}

Nach dem Satz von Hess kann die Reaktionsenthalpie aus den Standardbildungsenthalpien $\Delta H_\mathrm{f}$ berechnet werden:\\

\begin{equation} \label{GleichungHess}
\Delta H_\mathrm{V} = \sum_i \nu_i\Delta H_{\mathrm{f},i}\mathrm{(Produkte)} - \sum_i \nu_i\Delta H_{\mathrm{f},i}\mathrm{(Edukte)}
\end{equation}

Die Literaturwerte für die molaren Standardbildungsenthalpien sowie die nach Gleichung(\ref{GleichungHess}) berechnete Verbrennungsenthalpie für Naphtalin sind in Tabelle(\ref{TABELLELiteraturwerte Bildungsenthalpien}) aufgelistet:\\

\begin{table} \label{TABELLELiteraturwerte Bildungsenthalpien}
\caption{molare Standardbildungsenthalpien sowie die Verbrennungsenthalpie von Naphtalin}
\begin{tabular}{c|c}

  & $\Delta H_{\mathrm{f,m}}$ \\ 
 \hline 
 $\mathrm{C}_6\mathrm{H}_5\mathrm{COOH}$ & • \\ 
 \hline 
 $\mathrm{C}_{10}\mathrm{H}_8$ & • \\
 \hline 
 $\mathrm{O}_2$ & • \\ 
 \hline 
 $\mathrm{CO}_2$ & • \\ 
 \hline 
 $\mathrm{H}_2\mathrm{O}$ & • \\ 
 \end{tabular}  
\end{table}



\section{Literaturverzeichnis}

1,3,4\quad \emph{CRC Handbook of Chemistry and Physics}, 84. Auflage; D.R. Lide; CRC Press LLC: Boca Raton, \textbf{2004}.

2\quad Kabo, G.J.; Kozyro,A., A.; Frenkel, M.; Blokhin, A. V.\emph{Mol. Cryst. Liq. Cryst.}, \textbf{1999}, \emph{326}, 333-335.

5\quad Eckhold, Götz: \emph{Praktikum I zur Physikalischen Chemie}, Institut für Physikalische Chemie, Uni Göttingen, \textbf{2014}.

\vspace{0,5 cm}

6 \quad Eckhold, Götz: \emph{Statistische Thermodynamik}, Institut für Physikalische Chemie, Uni Göttingen, \textbf{2012}.

\vspace{0,5cm}

7 \quad Eckhold, Götz: \emph{Chemisches Gleichgewicht}, Institut für Physikalische Chemie, Uni Göttingen, \textbf{2015}.\\

\vspace{0,5cm}

8 \quad Atkins, P.W.: \emph{Physikalische Chemie}, Wiley-VCH, Weinheim, \
 textbf{2006}.\\

\vspace{0,5cm}

9 \quad Zemansky: \emph{Heat and Thermodynamics},Mc Graw-Hill, New York, \textbf{1990}.\\

\end{document}
