\documentclass[a4paper,12pt,oneside,onecolum,final,openany]{report} 
\usepackage[utf8x]{inputenc} 
\usepackage[T1]{fontenc} 
\usepackage[ngerman]{babel} 
\usepackage{amsfonts} 
\usepackage{amsmath} 
\usepackage{tabularx}
\usepackage[verbose]{placeins}
\usepackage{float}
\usepackage{caption}
\usepackage{subcaption}
\usepackage{pdfpages}
\usepackage{graphicx} 
\usepackage{amssymb} 
\usepackage[square,numbers]{natbib}
\usepackage{pgfplots}
\newcolumntype{R}[1]{>{\raggedleft\arraybackslash}p{#1}}
\newcolumntype{C}[1]{>{\centering\arraybackslash}p{#1}}
\title{2. Interferenz und Wellenlängenmessung} 
\author{Isaac-Johannes Maksso und Weiyi Zhu} 
\date{\today} 
\begin{document}
\section*{Messprotokoll V6}

Isaac Maksso, Julia Stachowiak\\
08.12.2016\\
Assistent: Jannis\\

organische Substanz:\\
$M(C_6H_5COOH)=$ 122,12 g$\cdot$mol$^{-1}$\\
$M(C_{10}H_8)=$ 128,17 g$\cdot$mol$^{-1}$\\
$p$(Bombe) (soll:20-25 atm)\\


\begin{tabular}{|c|c|c|c|}
\hline 
Versuch & $m$(Ni-Draht) & $m$(Benzoesäure)(soll:0,5-0,7 g) & $m$ (Substanz) \\ 
\hline 
1(Naphtalin) &&& \\ 
\hline 
2(Naphtalin) &&& \\ 
\hline 
(Naphtalin)  &&& \\ 
\hline 
4(org. Substanz)  &&& \\ 
\hline 
5(org. Substanz)  &&& \\ 
\hline 
6(org. Substanz)  &&& \\ 
\hline 
\end{tabular} 

\vspace{4cm}

Gerätefehler:\\
\vspace{3cm}

systematische Fehler:\\
\noindent
\end{document}