\documentclass[a4paper,12pt,oneside,onecolum,final,openany]{report} 
\usepackage[utf8x]{inputenc} 
\usepackage[T1]{fontenc} 
\usepackage[ngerman]{babel} 
\usepackage{amsfonts} 
\usepackage{amsmath} 
\usepackage{tabularx}
\usepackage[verbose]{placeins}
\usepackage{float}
\usepackage{caption}
\usepackage{subcaption}
\usepackage{pdfpages}
\usepackage{graphicx} 
\usepackage{amssymb} 
\usepackage[square,numbers]{natbib}
\usepackage{pgfplots}
\newcolumntype{R}[1]{>{\raggedleft\arraybackslash}p{#1}}
\newcolumntype{C}[1]{>{\centering\arraybackslash}p{#1}}
\title{2. Interferenz und Wellenlängenmessung} 
\author{Isaac-Johannes Maksso und Weiyi Zhu} 
\date{\today} 
\begin{document}
\section{Messprotokoll V6}

Isaac Maksso, Julia Stachowiak\\
08.12.2016\\
Assistent: Jannis\\
\\

$M(C_6H_5COOH)=$ 122,12 g$\cdot$mol$^{-1}$
$M(C_{10}H_8)=$ 128,17 g$\cdot$mol$^{-1}$



\textbf{Versuch 1}\\
$m$(Ni-Draht) :\\
$m$(Benzoesäure)(soll:0,5-0,7 g):\\
%$p_0$(Bombe) (soll:20-25 atm):\\
%$p_1$(Bombe):\\




\begin{table}[h!]
\begin{tabular}{l|c}
Temperatur/ °C& EMK/ V \\
\hline
 &\\
\hline
 &\\
\hline
 &\\
\hline
 &\\
\hline
 &\\
 \hline
 &\\
\hline
 &\\
\hline
 &\\
\hline
 &\\
\hline
 &\\
 \hline
  &\\
\hline
 &\\
\hline
 &\\
\hline
 &\\
\hline
 &\\
 \hline
  &\\
\hline
 &\\
\hline
 &\\
\hline
 &\\
\hline
 &\\
\hline
& \\ 
 \end{tabular}
\end{table}
\FloatBarrier 
\noindent
\end{document}